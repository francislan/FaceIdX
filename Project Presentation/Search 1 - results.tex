\documentclass[12pt]{article}
\usepackage[english]{babel}
\usepackage[utf8x]{inputenc}
\usepackage{amsmath}
\usepackage{amssymb}
\usepackage{graphicx}
\usepackage[a4paper]{geometry}
\usepackage{subfigure}
\usepackage{empheq}
\usepackage{stmaryrd} % for \llbracket
%\usepackage{fullpage}

%\usepackage{array}
%\usepackage{multirow}
%\usepackage{fancyhdr}
%\usepackage{tabularx}

% For plots
%\usepackage{tikz}
%\usepackage{pgfplots}

% For algorithms
%\@[algoruled]{algorithm2e}

\title{GPU-accelerated Face Identification using Principal Component Analysis}
\date{}
\author{Xianglu \textsc{Kong} (xk2122)\quad \quad \quad \quad \quad  Francis \textsc{Lan} (fl2381)}
\begin{document}

\maketitle

\section{Problem Formation}
Face identification is widely used in many situations, for surveillance and authentication purposes. With the growing size of datasets, fast computation is required to produce real-time performance.

\section{Motivation}
This problem is well fit for GPU. In the training phase, covariance computation, eigenspace projections are time consuming; In the testing phase, the euclidean distance needs to be heavily computed. Indeed, Principal Component Analysis (PCA) algorithms use a lot of linear algebra, which can be highly paralleled. Moreover, other image operations such as image resizing, grey scale converting can be optimized using GPU as well.

\section{Previous attempts}
Some implementations:
\begin{itemize}
\item Ashraf, Numaan. ``CUDA accelerated face recognition." NeST-NVIDIA Center for GPU Computing NeST, India (1995).
\item Kawale, Manik R., Yogesh Bhadke, and Vandana Inamdar. ``Parallel implementation of eigenface on CUDA." Advances in Engineering and Technology Research (ICAETR), 2014 International Conference on. IEEE, 2014.
\item Mateo, Julio Camarero. ``CPU/GPGPU/HW comparison of an Eigenfaces face recognition system." (2014).
\end{itemize}

\section{Our work}

\begin{figure}[h]
\centering
\includegraphics[width=\linewidth]{image_recognition_flow}
\caption{\bf Processing flow of face recognition} 
\label{fig1}
\end{figure}

The processing flow is represented in Fig\ref{fig1} (extracted from Mateo, Julio Camarero. ``CPU/GPGPU/HW comparison of an Eigenfaces face recognition system.")\\\\
We assume the input images are already well-cropped and we will mainly focus on the following steps:
\begin{enumerate}
\item Image normalization
\item Eigenface computation
\item Test image comparison
\end{enumerate}
Since the first step is not specific to PCA, we will not spend a lot of time and energy into it. We may use some existing implementations that show good performance.
\end{document}